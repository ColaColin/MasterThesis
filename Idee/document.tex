% Dokumentenklasse fuer Artikel waehlen
\documentclass[12pt,onecolumn,oneside,titlepage]{article}

% Deutsche Grundeinstellungen und DIN A4
\usepackage{a4}

% Zur Einbindung von Grafiken mit \includegraphics
\usepackage[pdftex]{graphicx,color}
\DeclareGraphicsExtensions{.pdf,.jpg,.png,.eps,.ps}
\usepackage{graphicx}
\graphicspath{ {images/} }

\usepackage{textcomp,upquote,lmodern,listings}
\usepackage{subcaption}
\usepackage{float}

% Bibliographie mit BibTeX
%\usepackage{natbib}

% Mehrsprachige Bibliographie mit babelbib

\usepackage[english]{babel}
%\usepackage[ngerman]{babel}

\usepackage{babelbib}

% Korrekte Umsetzung von Umlauten
\usepackage[utf8]{inputenc}
\usepackage{times}

\usepackage{url}
\usepackage{hyperref}

% Mathematische Symbole
\usepackage[intlimits,centertags]{amsmath}
\usepackage{amsfonts}
\usepackage{amssymb}
\usepackage{amsmath}

\DeclareMathOperator*{\argmax}{arg\,max}
\DeclareMathOperator*{\argmin}{arg\,min}

% Einrueckung der ersten Zeile eines Absatzes
\setlength{\parindent}{0em}

% Abstand zwischen Absaetzen
\setlength{\parskip}{1.5ex plus0.5ex minus0.5ex}

% Seitenstil
\pagestyle{headings}

% Seitennummerierung
\pagenumbering{arabic}
\setcounter{page}{2}

% Silbentrennungsliste
\hyphenation{native-Hello}

% -- Dokumentenbegin --------

\begin{document}

\title{\bfseries {\Large Proposal Masterthesis}\\[2cm]
      Investigation of possible improvements to increase the efficiency of the AlphaZero algorithm.}

\author{\textsc{Colin Clausen}\\[1cm]
        7.12.2019\\[3cm]}

\date{Dozent: \textsc{Prof. Dr.-Ing. Sven Tomforde}}

% -- Titelseite und Inhaltsverzeichnis erzeugen --------

\maketitle
\setcounter{tocdepth}{3}  % max. Tiefe im Inhaltsverzeichnis
\tableofcontents
%\listoffigures  % optionales Abbildungsverzeichnis
\newpage

\section{AlphaZero basics}

The AlphaZero algorithm \cite{silver2018general} is a simplification of AlphaGoZero \cite{silver2017mastering} and AlphaGo \cite{silver2016mastering}. AlphaGo gained fame for beating top human players at the game of Go, a feat previously thought still a decade away.
AlphaGo used a complicated system involving initialization with example games, random roleouts during tree searches and used multiple networks for different tasks. AlphaGoZero drastically simplified the system by only using a single network for all tasks,
and not doing any roleouts anymore, instead the network evaluation for the given positions is directly used.

The difference between AlphaGoZero and AlphaZero is mainly that AlphaGoZero involved comparing the currently trained network against the previously known best player by letting them play a set of evaluation games against each other.
Only the best player was used to generate new games. AlphaZero skips this and just always uses the current network to produce new games, surprisingly this appears to not give any disadvantage, learning remains stable.

The main advantage of the ``Zero`` versions is that they do not require any human knowledge about the game apart from the rules, the networks are trained from scratch by self-play alone.
This allows the algorithm to find the best way to play without human bias which seems to slightly increase final playing strength.
Additionally it allows to use the algorithm for research of games for which no human experts exist, such as No-Castling Chess \cite{NoCastleChess}.

\subsection{The AlphaZero algorithm}

The algorithm, as described in \cite{silver2018general}, has the goal of generating training examples to be able to train a neural network to play at extremely high level. 
Therefore at the center of learning stands a deep neural network which, given a board position, produces a policy and a value.
The policy describes move likelihoods, the value the chance of winning the game from the given board position.

The main idea is to produce examples of how to play better than the network by guiding a monte carlo tree search with the network. This averages out many position evaluation of the network over possible futures of a given position, producing a new higher quality policy for
the position.

By a work written in parallel to the AlphaGoZero paper this is described as thinking fast, via the network, and slow, 
via the search and compared to how humans at first have to think actively about a new task to do well, but with practice may learn to do well unconsciously without active thinking \cite{anthony2017thinking}.

The MCTS algorithm stores for every state-action pair $(s,a)$ in the tree a tuple of statistics, $\{N(s,a), W(s,a), Q(s,a), P(s,a)\}$, with $N(s,a)$ being the visit count of that action on that state,
$W(s,a)$ being the total action value, $Q(s,a)$ beeing the average action value and finally $P(s,a)$ being the prior probability of select $a$ in $s$, which is the network prediction.
To decide on a better policy for a given situation a number of simulations is executed. Every simulation starts at the root node, picking actions to go down the tree until a leaf is discovered, at which point the network is invoked to find the prior probability
for that leaf and backpropagate the value estimation by the network back up to the root, updating the node statistics on the way.

The action to be played on a given node $s$ is selected according to 

\begin{equation}
 argmax_a(Q(s, a) + U(s, a))
\end{equation}

where
\begin{equation}
 U(s,a) = C_{puct} P(s,a) \frac{\sqrt{N(s)}}{(1+N(s,a))}
\end{equation}

with $N(s)$ beeing the parent visit count and $C_{puct}$ beeing a constant controlling exploration.
Once a leaf is encountered the value predicted by the network is backpropagated upwards, adding to $N(s,a)$ and $W(s,a)$ of nodes and recalculating the average action values.

The final policy generated by MCTS is determined by the visitation count of the actions of the root node.

Using this tree search games are played out, the resulting game states are stored with the MCTS policy and the final game result. The network is then trained to predict the MCTS policy for the states and the final result of the game.
Since the policy generated by the MCTS is always better than the one by the network alone this improves the playing strength of the network, which in the next iteration will allow MCTS to produce even better examples of play.

\section{Previous work}

Understanding previous work is not only important to understand where to search for further improvements without reinventing the wheel, but also to speed up the learning progress to be able to experiment faster on proposed ideas.

\subsection{Network and training modifications}

The original network used in AlphaZero is a tower of 20 or 40 residual network blocks with 256 convolutional filters. Some ways have been found to improve this network to speed up training without any actual change to the AlphaZero algorithm, mainly by applying progress in the artificial neural network design.

\begin{itemize}
 \item Enhancing the blocks with Squeeze-and-excitation elements \cite{hu2018squeeze}, has been proposed by the Leela Chess Zero projects, which reimplements AlphaZero for Chess as a distributed effort \cite{leela0sq}.
       A very similar approach is used in \cite{wu2019accelerating} which also cites the similarity to Squeeze-and-excitation networks.
 \item The original design of the network only uses 2 layers in the policy head and a single filter in the value head, using 32 filters has been found to speed up training, as reported by \cite{oracledevs6} based on earlier work done by the Leela Chess Zero project.
 \item Cyclic learning rates can be used to improve the network fitting to the data, \cite{oracledevs6} shows a modest speed up.
\end{itemize}

\subsection{Modification of the tree search}

The behavior of the tree search can be modified to change how available time is spent to understand given game positions.

\begin{itemize}
 \item Instead of using a single network \cite{lan2019multiple} proposes to combine multiple networks of different sizes, especially two networks, one big, one small. Specifically, in the paper, the 
 big network has 10 residual blocks with 128 convolutional filters, whereas the small network has 5 residual block with 64 convolutional filters. This results in one forward pass of the big network taking as long as eight forward passes of the small network.
 Using the small network in most positions reduces computational requirements, 
 which allows more search steps. This appears to increase playing strength given the same computational budget, the authors state that training is accelerated by a factor of at least 2.
 \item An idea called Playout Caps is proposed in \cite{wu2019accelerating}. They drastically reduce the number of MCTS playouts randomly in most of the moves played. It allows to play more games in the same time, somewhat similar to 
 the concept of using a combination of a small and a big network, the authors argue that this gives more data to train the value target, which is starved for data since every game played is only a single data point for this target. A training acceleration of 27\% is stated.
 \item Propagation of terminal moves through the search tree to simplify the search is proposed by the Leela Chess Zero project, with a moderate improvement in playing strength \cite{leela0propagation}. This kind of improvement falls
 into a family of various other ways to improve the handling of the search tree, such as detecting positional transpositions. From the AlphaZero paper it is not clear to what degree DeepMind used such optimizations.
 \item The Leela Chess Zero project suggests analyzing statistics, namly the Kullback-Leibler divergence of the policy as it evolves during the tree search, to understand how complex a position is and apply more tree search evaluations on complex situations. They find a noteable increase
   in playing strength \cite{leela0kldgain} using the same computational budget.
\end{itemize}



\subsection{Learning target modifications}

\begin{itemize}
 \item \cite{wu2019accelerating} Proposes to predict the opponent's reply to regularize training. A modest improvement is shown.
 \item \cite{wu2019accelerating} also shows major improvements can be made if some domain specific targets are used to regularize the learning process. This hints at the possibility to search for ways to automatically determine such regularization targets.
 They again point at how the learning is highly constrained by available data on the value target and how these additional targets may help alleviate this.
 \item Forced Playouts and Policy Target Pruning, proposed in \cite{wu2019accelerating}, force that nodes, if they are ever selected, receive a minimum number of further playouts. Pruning is used to remove this from the policy distribution used to train the network for bad moves,
 as the forced playouts are only meant to improve exploration. Training progress is stated to be accelerated by 20\%.
 \item The Leela Chess Zero project uses a modification of the value target, which explicitly predicts a drawing probability, as this allows the network to tell the difference between a very uncertain position and a position that is very likely drawn \cite{leela0wdl}.
 \item \cite{anonymous2020threehead} suggests using a third output head which predicts the win probability for every move. This can be used to shortcut the tree search, reducing the number of network evaluations, but the win probability estimate might be of a worse quality.
  Small improvements are claimed.
 \item Instead of using the result of games as a learning target for the value output of the network \cite{oracledevs6} proposes to use the average value of the MCTS node at the end of the MCTS search for a given position. This has the advantage of providing richer data,
 and reducing the influence of a single bad move at the end of a game, which would taint the evaluation of all positions in that game. However, since this MCTS-based value has other accuracy issues they specifically propose to combine the two values, which shows a noticeable improvement.
\end{itemize}


\subsection{Training data enhancements}
\begin{itemize}
 \item There can be multiple copies of the same identical position in the training data. \cite{oracledevs6} shows that averaging the targets for these positions and reducing them to a single example is beneficial. This is shown on the game of Connect 4,
 which is especially prone to identical positions showing up, so it is not clear how well it might translate to more complex games.
 \item As noticed in \cite{oracledevs6}, the playing strength quickly increases the moment the very first training examples are removed. This is likely because those examples were effectively generated with a random network and are thus very bad, holding back training.
 A modification to the training data windowing is proposed, which fixes this.
\end{itemize}

\section{Motivation}

The AlphaZero algorithm is a way to attain super human playing strength in arbitrary board games, even very complex ones like Go.
At least for more complex games like Go this is however still only possible by using very substantial amounts of processing power, thousands of machines clustered together,
making it impossible to leverage the algorithm on hard problems for anyone but the largest organizations.

This motivates the search for improvements that allow to achieve results with lower requirements on computational power, preferably without losing the generality of AlphaZero.
Most time during AlphaZero training is spent on generating example games,
thus it follows that to improve learning speed either more learning has to happen per game or games have to be played out faster.

Previous work has established that large efficiency gains are possible and further improvements shall be systematically investigated and compared to previous improvements as well as a the baseline algorithm to understand how the learning progress is affected and if efficiency gains
are accumulative with other known enhancements.
Additionally it shall be attempted to find metrics by which to measure the effects of the modifications beyond simple playing strength.

\section{Goals}

The general plan is as follows, I will describe the steps in-depth in the following subsections.

\begin{enumerate}
 \item Implement a clean version of the algorithm, taking special care to modularize the system as much as possible to be able to quickly switch on or off modifications to the base algorithm.
 \item Specify metrics to be used to evaluate proposed improvements as well as metrics that can give a deeper understanding of the training process.
 \item Implement previously published improvements, at least ones that are not too complicated. This has the advantage of speeding up experiments, as well as makes sure that proposed improvements are tested to improve performance on top of known enhancements.
 \item Implement specific ideas to improve the algorithm beyond the published ones and evaluate them, based on the metrics defined before.
\end{enumerate}


\subsection{Means of evaluation}

To be able to experiment fast without massive usage of computing resources simpler games are typically used when evaluating AlphaZero.
Previous work \cite{oracledevs} has shown that Connect 4 can be 99\% solved using AlphaZero, since the game can be perfectly solved using classical approaches this can be cleanly verified.
The simplicity of Connect 4 makes it possible to experiment faster on variations of AlphaZero, which is why I plan to use training speeds on Connect 4 to a 99\% fit to perfect play as my primary metric of evaluation. In case
experiments need to be quicker than anticipated it would also be possible to slightly reduce the size of the board played on.

A remaining problem is how to measure the training time. Using wall clock time requires all experiments to be run on the exact same hardware with no background loads. This would restrict resources.
The number of generated games cannot be used, as some improvements might reduce the number of games played, but increase computational time spent on a single game.

I suggest to run a baseline version of AlphaZero on hardware used for an experiment first and measure the moves played per second over a timeframe of a few minutes. Then the time used for a training run on that hardware can be multipled with the 
measured moves per second to get a normalized time measurement, that considers how fast the hardware used for the experiment was.

I conjecture this is a valid way of comparing the learning efficiency of the algorithm between runs made on different hardware.
To verify this conjecture multiple runs of the baseline algorithm can be done on the same hardware, measuring the time deviation between different runs.
Then multiple runs can be made with different hardware each time, comparing the deviation between the normalized times between them.
The deviation of the normalized times with different hardware each run should not be much higher than the deviation of unnormalized times with the same hardware for every training run.


\subsection{Understanding the learning process}

An interesting property of the AlphaZero algorithm is the perceived stability of the training. Other learning systems often struggle to learn two players games in a pure self-play regime, because they get trapped in an endless cycle of learning and unlearning the same mistakes.

A typical way to prevent this normally would be to include earlier iterations of a learner in the learning process, to prevent forgetting previous knowledge.
This was originally done in AlphaGo, however with further work it became clear that even in the AlphaZero regime, with no measures taken at all to prevent forgetting learning still works fine.
This raises the question of how stable the training truly is. It might be that some amount of forgetting still occurs, just not enough to stop progress.

To research this a dataset of specific game situations could be used, checking which positions are played correctly after every learning iteration. This way it can be compared how every learning iterations adds or removes more understanding of the game to the used neural network.
A better understanding of the learning progress could be valuable to suggest additional ways to speed up training, for example in regards to how the training data window is currently managed.

Further statistics could be collected on the training progress, such has the variation of positions encountered, win/loss rates between the first and second player, draw rates or various statistics on the behavior of the tree search.

Specifically I want to ask the questions: ''Can a forgetting and relearning cycle in the learning process be found using a fixed dataset of Connect 4 positions? What statistics have to be tracked for this? ``

\subsection{Specific improvements to be investigated}

As a last step of my work on the AlphaZero algorithm I want to implement some specific ideas and evaluate them for their potential of increasing learning efficiency.

\subsubsection{Network modifications}

Further advances in research on artificial neural networks could be applied, such as Gather-Excite networks \cite{DBLP:journals/corr/abs-1810-12348} or the especially resource efficient network blocks of mobilenetv3 \cite{howard2019searching}.
This does not constitute a major advance in the AlphaZero algorithm, but may still increase efficiency of the learning system and is thus worth at least a short investigation.

\subsubsection{Playing games as trees}

Various previous work points at how the value target is starved for data, since it only receives examples by whole games that can be tarnished by random mistakes in the last few moves and suggest improvements targetting this \cite{wu2019accelerating}, \cite{oracledevs6}, \cite{lan2019multiple}.

I suggest to occasionally copy game states and playing on differently, for example using the 2nd best move found, in the copy.
This will result in early game states having multiple possible continuations, which are played out to the end. Thus early game states played will tend to have multiple known outcomes,
which can be averaged to produce a more accurate estimate of the value of those positions. The influence of a single bad move towards the end should be reduced and the value head of the network should learn faster, since the training data will be of higher quality.
There should be no substantial computation cost to implement this, as it just changes the structure in which games are played.

\subsubsection{Automatic ways to find auxilary targets}

\cite{wu2019accelerating} shows that using domain specific features as auxilary targets to regularize learning is very helpful. This motivates the search for an automatic way to find such targets.

I propose two means of finding such targets.

Evolved targets could be created by an evolutionary algorithm tasked with evolving some form of query string that creates features which are especially useful to train a small convolutional network to predict the final value of the respective game.
A likely problem with this approach however might be the possibly substantial cost of this evolutionary process.

To reduce the additional computational cost my second proposal to find regulization targets is to use the internal features the network learns. These are learnt anyway and are thus free.
Specifically I propose to designate a layer $\mathbb{F}$ in the network and connect an additional head $\mathbb{H}_{future}$ to the layer before $\mathbb{F}$. 
The network stem $\mathbb{S}$ up to $\mathbb{F}$ is then used to produce additional learning targets by applying $\mathbb{S}$ to the 2-ply future of every game state used as training data.
This uses the encoding the network found to be meaningful for game playing to encode the immediate future of the game and uses this as a learning target. The usage of the 2-ply future is required, since otherwise the network would just be tasked with producing
its own internal features. Instead the network will be required to predict the future game situation after one action by each player.

The reason for attaching the head $\mathbb{H}_{future}$ before $\mathbb{F}$ is such that $\mathbb{F}$ only contains an encoding relevant to the value and policy heads, not an encoding that also tries to encode information for $\mathbb{H}_{future}$ of previous iterations.
It might be worthwhile to experiment with both options: Adding $\mathbb{H}_{future}$ before $\mathbb{F}$ or after $\mathbb{F}$.
Another avenue of experiments could be to try n-ply futures, for example 4-ply or 6-ply to evaluate how predicting a more distance future could be helpful.

\subsubsection{Using the self-playing phase as an evolutionary process}

There are many hyperparameters involved in AlphaZero which have to be tuned. Due to the general nature of how to algorithm is designed around the network learning to imitate the MCTS results one can argue that the goal of
nearly all hyperparameters in AlphaZero should be to increase playing strength of the MCTS. Searching optimal values for the hyperparameters however is extremely expensive and it might be that depending on the learning progress of the network 
the best hyperparameters are not actually constant, but shift during training.

I propose to let multiple instances of players play games against each other during self play, with small differences in their hyperparameters. The generated games should still be valid as training data, but as an additional benefit, at no computational cost,
the results of the games can be used to judge which hyperparameters appear to provide superior playing strength. After some number of games are played between the different players the worst players can be eliminated and new players are introduced to find even better hyperparameters.
The relative results of the players could be tracked using a rating system between the players such as Elo or Glicko.

This allows the learning process to continuously pick the best hyperparameters and also adjust them if necessary after some training process has been made.


% -- Literaturverzeichnis --------

\bibliographystyle{plain}     % nummeriere Zitate [1], [2], ...

% Quellenangaben stehen in einer separaten BibTeX-Datei Seminararbeit.bib
\bibliography{document}

\end{document}

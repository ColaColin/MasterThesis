% Dokumentenklasse fuer Artikel waehlen
\documentclass[12pt,onecolumn,oneside,titlepage]{article}

% Deutsche Grundeinstellungen und DIN A4
\usepackage{a4}

% Zur Einbindung von Grafiken mit \includegraphics
\usepackage[pdftex]{graphicx,color}
\DeclareGraphicsExtensions{.pdf,.jpg,.png,.eps,.ps}
\usepackage{graphicx}
\graphicspath{ {images/} }

\usepackage{textcomp,upquote,lmodern,listings}
\usepackage{subcaption}
\usepackage{float}

% Bibliographie mit BibTeX
%\usepackage{natbib}

% Mehrsprachige Bibliographie mit babelbib

\usepackage[english]{babel}
%\usepackage[ngerman]{babel}

\usepackage{babelbib}

% Korrekte Umsetzung von Umlauten
\usepackage[utf8]{inputenc}
\usepackage{times}

\usepackage{url}
\usepackage{hyperref}

% Mathematische Symbole
\usepackage[intlimits,centertags]{amsmath}
\usepackage{amsfonts}
\usepackage{amssymb}
\usepackage{amsmath}

\DeclareMathOperator*{\argmax}{arg\,max}
\DeclareMathOperator*{\argmin}{arg\,min}

% Einrueckung der ersten Zeile eines Absatzes
\setlength{\parindent}{0em}

% Abstand zwischen Absaetzen
\setlength{\parskip}{1.5ex plus0.5ex minus0.5ex}

% Seitenstil
% \pagestyle{headings}

% Seitennummerierung
\pagenumbering{arabic}
\setcounter{page}{2}

% Silbentrennungsliste
\hyphenation{native-Hello}

% -- Dokumentenbegin --------

\begin{document}
%%%%%%%%%%%%%%%%%%%%%%%%%%%%%%%%%%%%%%%%%%%%%%%%%%%%%%%%%%%%%%%%%%
%%                                                              %%
%%                          Titelseite                          %%
%%                                                              %%
%%%%%%%%%%%%%%%%%%%%%%%%%%%%%%%%%%%%%%%%%%%%%%%%%%%%%%%%%%%%%%%%%%

\begin{center}
{\huge \it Masterthesis}

\thispagestyle{empty}

\vspace{2cm}

{\Large \bf Investigation of possible improvements to increase the efficiency of the AlphaZero algorithm.}

\vspace{2.25cm}

\vspace{2.25cm}

{\large 
Christian-Albrechts-Universität zu Kiel \\
Institut für Informatik  \\
}

\end{center}

\vspace{2cm}

\begin{tabular}{ll}
angefertigt von:             & {\bf Colin Clausen} \\
betreuender Hochschullehrer: & Prof. Dr.-Ing. Sven Tomforde \\%
\end{tabular}

\vspace{1cm}

\begin{center}
Kiel, 20.7.2020
\end{center}


\pagebreak

\newpage\null\thispagestyle{empty}\newpage


%%%%%%%%%%%%%%%%%%%%%%%%%%%%%%%%%%%%%%%%%%%%%%%%%%%%%%%%%%%%%%%%%%
%%                                                              %%
%%                Selbstständigkeitserklärung                   %%
%%                                                              %%
%%%%%%%%%%%%%%%%%%%%%%%%%%%%%%%%%%%%%%%%%%%%%%%%%%%%%%%%%%%%%%%%%%
\noindent {\bf Selbstständigkeitserklärung}

\vspace{1.5cm}

\noindent Ich erkläre hiermit, dass ich die vorliegende Arbeit selbstständig und nur unter Verwendung der angegebenen Literatur und Hilfsmittel angefertigt habe.

\vspace{2cm}
\noindent ............................................................... \\
Colin Clausen

\thispagestyle{empty}

\pagebreak

\newpage\null\thispagestyle{empty}\newpage


\tableofcontents

\pagebreak



\section{Introduction}


\section{Previous work}

\subsection{Monte Carlo Tree Search}

\subsection{AlphaZero}

\subsection{Extensions to AlphaZero}


\section{Evaluated novel improvements}
\subsection{Network modifications}
\subsection{Playing games as trees}
\subsection{Automatic auxilary features}

\section{Experiments}

\subsection{Baselines}
\subsubsection{AlphaZero implementation}
\subsubsection{Extended AlphaZero}


\subsection{Results on novel improvements}
\subsubsection{Network modifications}
\subsubsection{Playing games as trees}
\subsubsection{Automatic auxilary features}
\subsubsection{Evolutionary hyperparameters}






\pagebreak

\cite{silver2018general}

% -- Literaturverzeichnis --------

\bibliographystyle{plain}     % nummeriere Zitate [1], [2], ...

% Quellenangaben stehen in einer separaten BibTeX-Datei Seminararbeit.bib
\bibliography{document}

\end{document}
